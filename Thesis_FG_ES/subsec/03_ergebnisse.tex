\subsection{Ergebnisse und Probleme}
\label{subsec:03ergebnisse}
Die erste Versuchsreihe wird mit dem Fahrzeug auf dem Tisch platziert durchgef�hrt. Hierbei soll ausschlie�lich die Personendetektion und das Tracking ohne Bewegung des Fahrzeugs getestet werden. Die Resultat sind durchaus vielversprechend. Die Personverfolgung aus Detektion und Tracking k�nnen die Person in diesen Versuchen verl�sslich verfolgen, wobei auch fehlerhaftes Tracking erkannt wurde. Das Erkennen des misslungenen Trackings ist hingegen nicht besonders robust und somit noch ausbauf�hig. \\
In der zweiten Versuchsreihe wird die vollst�ndige Personenverfolgung samt Folgeregelung getestet. Um �berhaupt eine Person im Kamerabild zu erhalten, muss hierf�r die Kamera stark nach oben geneigt werden. Diese ver�nderte Perspektive stellt sich als Problem f�r die Personenverfolgung dar, da das Bild nun stark verzerrt wird. Sowohl die Detektion als auch das Tracking verschlechtert sich aus dieser Perspektive und sind deutlich weniger robust. Dennoch gelingt das Folgen einer Person regelm��ig f�r eine l�ngere Zeitdauer. Um die Ergebnisse in Zukunft deutlich zu verbessern, sollte eine perspektivische Transformation des Kamerabildes vor der Personenverfolgung durchgef�hrt werden. \\
Ein weiteres Problem ist die Verfolgung einer Personen um enge Ecken, da der fest eingestellte Abstand von zwei Metern f�r diesen Fall zu gro� ist. Hierf�r w�ren zwei L�sungsans�tze m�glich. Einerseits k�nnte der Abstand variabel gestaltet werden, sodass dieser bei der Fahrt um eine Ecke verringert wird. Dabei entsteht das Problem, dass die Person nicht mehr komplett im Kamerabild erfasst werden kann. Somit w�re die optimale L�sung, die Verwendung eines Kalman Filters, der die Person mithilfe der zuvor berechneten Bewegung auch au�erhalb des Kamerasichtfeldes weiter verfolgen kann. Dieses k�nnte in Kombination zum bereits implementierten Tracker eingesetzt werden und somit die gesamte Personenverfolgung verbessern.
