\subsection{Ergebnisse und Probleme}
\label{subsec:03ergebnisse}
Die erste Versuchsreihe wurde mit dem Fahrzeug auf dem Tisch platziert durchgef�hrt. Hierbei sollte ausschlie�lich die Personendetektion und das Tracking ohne Bewegung des Fahrzeugs getestet werden. Die Resultate waren durchaus vielversprechend. Die Personverfolgung aus Detektion und Tracking konnte die Person in diesen Versuchen verl�sslich verfolgen, wobei auch fehlerhaftes Tracking erkannt wurde. Das Erkennen des misslungenen Trackings war hingegen nicht besonders robust und somit noch ausbauf�hig. \\
In der zweiten Testphase wurde die vollst�ndige Personenverfolgung samt Folgeregelung getestet. Damit eine Person �berhaupt im Sichtfeld der Kamera enthalten war, musste hierf�r die Kamera stark nach oben geneigt werden. Diese ver�nderte Perspektive stellte sich als Problem f�r die Personenverfolgung heraus, da das Bild nun stark verzerrt war. Sowohl die Detektion als auch das Tracking verschlechterten sich aus dieser Perspektive und waren deutlich weniger robust. Dennoch gelang das Folgen einer Person regelm��ig f�r eine l�ngere Zeitdauer. Um die Ergebnisse in Zukunft deutlich zu verbessern, w�re eine perspektivische Transformation des Kamerabildes vor der Durchf�hrung der Personenverfolgung zu empfehlen. \\
Ein weiteres Problem ist die Verfolgung einer Personen um enge Ecken, da der fest eingestellte Abstand von zwei Metern f�r diesen Fall zu gro� ist. Hierf�r w�ren zwei L�sungsans�tze m�glich. Einerseits k�nnte der Abstand variabel gestaltet werden, sodass dieser bei der Fahrt um eine Ecke verringert wird. Dabei entsteht jedoch das Problem, dass die Person nicht mehr komplett im Kamerabild erfasst werden kann. Somit w�re die optimale L�sung, die Verwendung eines Kalman Filters, der die Person mithilfe der zuvor berechneten Bewegung auch au�erhalb des Kamerasichtfeldes weiter verfolgen kann. Dieses k�nnte in Kombination zum bereits implementierten Tracker eingesetzt werden und somit die gesamte Personenverfolgung verbessern.
