\subsection{Ergebnisse und Probleme}
\label{subsec:02ergebnisse}
Wie im Kapitel \ref{subsec:02odom} bereits erw\"ahnt, wurde das Extended Kalmanfilter nicht fertig eingestellt. Das implementierte EKF benutzt die IMU-Daten und die Geschwindigkeit, welche durch den Hall-Sensor berechnet wird. Obwohl die Y-Geschwindigkeit des Fahrzeugs nicht als Eingabegr\"o{\ss}e f\"ur das EKF verwendet wird und nur die Steuerung der Geschwindigkeit in X-Richtung und des Yaw Winkels zugelassen sind, bewegt sich das Auto in den gefilterten Odometriedaten des EKF seitw\"arts. Das Einstellen der maximalen Be- und Entschleunigungswerte des Fahrzeugs hat zwar zur Verbesserung der Absch\"atzung beigetragen, das Problem der seitlichen Bewegung jedoch nicht behoben.