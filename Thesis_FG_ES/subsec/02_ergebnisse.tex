\subsection{Ergebnisse und Probleme}
\label{subsec:02ergebnisse}
Wie im Kapitel \ref{subsec:02odom} bereits erw\"ahnt, wurde das Extended Kalmanfilter nicht vollst\"andig eingestellt. Das implementierte EKF benutzt die IMU-Daten sowie die Geschwindigkeit, welche mit Hilfe des Hall-Sensors berechnet wird. Obwohl die $y$-Geschwindigkeit des Fahrzeugs nicht als Eingabegr\"o{\ss}e f\"ur das EKF verwendet wird und nur die Steuerung der Geschwindigkeit in $x$-Richtung und des Yaw Winkels zugelassen sind, bewegt sich das Fahrzeug in den gefilterten Odometriedaten des EKF seitw\"arts. Das Einstellen der maximalen Be- und Entschleunigungswerte des Fahrzeugs hat zwar zur Verbesserung der Absch\"atzung beigetragen, das Problem der seitlichen Bewegung jedoch nicht vollst\"andig behoben.\\
Obwohl der Navigation Stack und haupts\"achlich der TEB local planner insgesamt gut eingestellt sind, gibt es einige Probleme, die es in Zukunft zu l\"osen gilt.
\begin{itemize}
	\item \textbf{Artefakte Laserscan:} Das Package \texttt{depthimage\_to\_laserscan} liest eine Zeile aus dem Tiefenbild, welches mit Rauschen behaftet ist. Dadurch entstehen Artefakt-Punkte auf der Costmap, die der Planer als Hindernis erkennt, die allerdings nicht real sind. Obwohl das Tiefenbild bereits durch einen Medianfilter (Filtergr\"o\ss{}e 5) bearbeitet wird, tritt dieser Effekt immer noch auf. \\
	Ein L\"osungsansatz w\"are die Gr\"o\ss{}e des Filters zu erh\"ohen, allerdings ist der verwendete Filter aus dem Package \texttt{pses\_kinect\_utilities} mit Gr\"o\ss{}e 5 begrenzt. Ein weiterer L\"osungsansatz w\"are den Laserscan aus mehreren Zeilen des Tiefenbildes auszulesen und von jeder Spalte den maximalen Wert herauszuschreiben. 
	\item \textbf{Recovery:} Da die lokale Costmap nicht nur die aktuellen Daten aus dem Laserscan beinhaltet, sondern auch mit den vorangegangenen Daten eine Karte zusammenbaut, k\"onnen Artefakte mit in die Karte eingetragen werden. Passiert das h\"aufiger an einer Stelle, kann es den Weg f\"ur den Trajektorienplaner versperren.\\
	Um diese falsch detektierten Punkte wieder loszuwerden, gibt es den Parameter \texttt{recovery\_behaviors}, der verschiedene Szenarien zum Leeren des Costmap beinhaltet. Es gibt einige Parameter\footnote{http://wiki.ros.org/move\_base\#Parameters}, die den Zeitpunkt des Ausf\"uhrens des Recovery Programms beeinflussen, diese sollten eingestellt werden. Es sei angemerkt, dass der Parameter \texttt{clearing\_rotation\_allowed} auf false gesetzt sein sollte, da das Fahrzeug eine solche Bewegung nicht vollziehen kann.
	\item \textbf{R\"uckw\"arts fahren:} Sobald das Fahrzeug nah an einem Hindernis f\"ahrt, kommt es in einen Bereich in dem der Wert der Straffunktion hoch ist. Das bedeutet, dass das Fahrzeug sich in diesem Bereich nur langsam bewegen darf. Es hat sich gezeigt, dass in diesem Fall der TEB local planner nur noch vereinzelt Geschwindigkeitswerte ausgegeben hat und im Hindernis stecken geblieben ist.\\
	Der Navigation Stack sollte entsprechend eingestellt werden, dass dieser in solchen Situationen r\"uckw\"arts f\"ahrt und sich somit Platz verschafft, um am Hindernis vorbei zu fahren.
	\item \textbf{AMCL Fehldetektion:} AMCL dient dazu sich besser in einer Karte zu orientieren, indem es den Laserscan auf die Karte projiziert und damit die aktuelle Position bestimmt. W\"ahrend der Testfahrten kam es gelegentlich vor, dass AMCL in einer Kurve die hintere Wand auf die vordere Wand gemapt hat und somit den Weg verschlossen hat.\\
	Das Verhalten ist abh\"angig von der Qualit\"at der Fahrt und sollte untersucht werden. AMCL\footnote{http://wiki.ros.org/amcl} bringt selbst Parameter mit, die zum Verbessern dieses Verhaltens beitragen k\"onnten.
\end{itemize}