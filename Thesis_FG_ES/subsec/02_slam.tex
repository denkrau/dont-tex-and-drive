\subsection{SLAM}
\label{subsec:02slam}

Die Abk"urzung SLAM steht f"ur \emph{Simultaneous Localization And Mapping} und besteht f"ur einen Roboter darin, gleichzeitig eine Karte seiner Umgebung zu erstellen bzw. zu verbessern und sich darin zu lokalisieren. In ROS stehen mehrere Pakete zur Verf"ugung, welche SLAM implementieren, n"amlich \texttt{gmapping} und \texttt{hector\_slam}. Letzteres hat den Vorteil, dass es keine Odometriedaten ben"otigt, und nur anhand des Laserscans die Karte erstellt.\\
Um eine zuverl"assige Karte aufzubauen sind allerdings folgende Punkte zu beachten. Erstens muss das Auto langsam und m"oglichst geradeaus (ohne Schwingungen) fahren, um zu vermeiden, Artefakten in die Karte einzubauen. Die Verwendung der vom Fachgebiet bereitgestellten Pakete \texttt{pses\_dashboard} und \texttt{CarControl-App} kann sich beispielsweise zur Steuerung des Autos als n"utzlich erweisen. Falls \texttt{gmapping} benutzt wird, ist die Qualit"at der Karte und der Lokalisierung ebenso mit der Genauigkeit der Odometriedaten eng verbunden.\\
Das \texttt{hector\_slam}-Paket wurde zum Aufbau einer Karte des Stocks des Fachgebiets verwendet. Da die Qualit"at der Karte jedoch nicht zufriedenstellend war, wurde zur Lokalisierung die von den Tutoren vorgelegte Karte verwendet.