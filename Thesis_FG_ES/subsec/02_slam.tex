\subsection{SLAM}
\label{subsec:02slam}

Die Abk"urzung SLAM steht f"ur \emph{Simultaneous Localization And Mapping} und besteht f"ur einen Roboter darin, gleichzeitig eine Karte seiner Umgebung zu erstellen bzw. zu verbessern und sich darin zu lokalisieren. In ROS stehen mehrere Pakete zur Verf"ugung, welche SLAM implementieren, die bekanntesten sind \texttt{gmapping} und \texttt{hector\_slam}. Letzteres hat die Eigenschaft, dass es keine Odometriedaten ben"otigt und nur anhand des Laserscans die Karte erstellt.\\
Um eine realistische Karte aufzubauen sind mehrere Aspekte zu beachten. Das Fahrzeug sollte langsam und m"oglichst geradeaus (ohne Schwingungen) fahren, um zu vermeiden, dass Fehldetektionen in die Karte eingebaut werden. Die Verwendung der vom Fachgebiet bereitgestellten Pakete \texttt{pses\_dashboard} und \texttt{CarControl-App} k\"onnen zur Steuerung des Fahrzeugs genutzt werden. Falls \texttt{gmapping} benutzt wird, h\"angt die Qualit"at der Karte und Lokalisierung ebenso von der Genauigkeit der Odometriedaten ab.\\
Das \texttt{hector\_slam}-Paket wurde zum Aufbau einer Karte des Fachgebiets verwendet. Da die Qualit"at der Karte jedoch nicht zufriedenstellend war, wurde die von den Tutoren vorgelegte Karte verwendet.