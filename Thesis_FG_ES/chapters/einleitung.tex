\section{Einleitung}
\label{sec:einleitung}
\textbf{Im Projektseminar Echtzeitsysteme werden die Teilnehmer in Gruppen \'{a} 5 Leute aufgeteilt. Jede Gruppe bekommt ein Modellfahrzeug, welches mit Sensorik und Aktorik so ausgestattet ist, dass das Fahrzeug autonom fahren kann. Die Gruppen sollten mit dem Fahrzeug drei Aufgabenstellungen umsetzten, wobei nur die erste, das Fahren eines Rundkurses entlang einer Wand ohne Hindernisse, vorgegeben war. Die anderen beiden Aufgabenstellungen konnte man entweder aus einer Liste ausw\"ahlen, oder sich nach Absprache mit den Betreuern eine eigene Aufgabe ausdenken. Unsere Wahl f\"ur die zweite Aufgabe war es, einen Rundkurs mit Hindernissen zu absolvieren. Es war unser Ziel f\"ur die erste und zweite Aufgabenstellung denselben Ansatz zu verwenden. Des weiteren haben wir es uns zur Aufgabe gemacht, eine Personenverfolgung zu implementieren. Diese sollte eine Person mithilfe des Kamerabildes detektieren und ihr folgen, sobald sie sich bewegt. In diesem Dokument befindet sich sowohl die Beschreibung unserer Lösung f\"ur die jeweiligen Aufgabenstellungen, als auch eine Beschreibung der Probleme beim Umsetzten, ein Fazit und ein Ausblick auf Entwicklungen die wir gerne mit mehr Zeit noch get\"atigt h\"atten.}
