\section{Einleitung}
\label{sec:einleitung}
Im Rahmen des Projektseminars wurden drei Aufgaben zum Thema autonomes Fahren bearbeitet. Dabei geht es um die Kombination zwischen der Technik des Fahrzeugs und der dynamischen Regelung der Motor- und Lenksteuerung. Dazu ist das Fahrzeug mit diverser Sensorik ausgestattet, die dazu dient die Umgebung zu erfassen und daraus eine Trajektorie zu planen. \\
In der ersten Aufgabe geht es darum, einen Rundkurs ohne Hindernisse zu absolvieren. Dazu wird zuerst eine Regelung mit einem PD-Regler implementiert, der den seitlichen Ultraschall Sensor nutzt und den Abstand zur Wand regelt. Im Anschluss wird die Trajektorienplanung Navigation Stack aufgesetzt, welche die Daten des Tiefenbildes, eine Karte und die Odometrie des Fahrzeuges nutzt, um eine Trajektorie zum gesetzten Ziel zu berechnen.\\
Darauf aufbauend ist Aufgabe Zwei die Absolvierung eines Rundkurses, in dem unbekannte Hindernisse aufgebaut sind. Dazu wird ebenso der Navigation Stack verwendet. Die Hindernisse m\"ussen erkannt und in Kombination mit der globalen Karte eine sichere Trajektorie geplant werden. Genauere Informationen zu den Ans\"atzen und der Umsetzung der ersten beiden Aufgaben finden sich in Kapitel \ref{sec:trajektorienplanung}.\\
Das dritte Thema ist die Erkennung von Personen anhand des Bildes der Kinect2 Kamera und Hinterherfahren des jeweiligen Ziels. Dabei wird eine Person im Bild detektiert, die Entfernung zum Zielobjekt berechnet und mit einem PD-Regler dort hin navigiert. Eine genaue Beschreibung des Vorgehens findet sich in Kapitel \ref{sec:personenverfolgung}.