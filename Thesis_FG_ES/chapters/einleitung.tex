\section{Einleitung}
\label{sec:einleitung}
\textbf{Im Projektseminar Echtzeitsysteme werden die Teilnehmer in Gruppen \'{a} 5 Leute aufgeteilt. Jede Gruppe bekommt ein Modellfahrzeug, welches mit Sensorik und Aktorik so ausgestattet ist, dass das Fahrzeug autonom fahren kann. Jede Gruppe sollte mit dem Fahrzeug drei Aufgabenstellungen umsetzten, wobei nur die erste, das Fahren eines Rundkurses entlang einer Wand ohne Hindernisse, vorgegeben war. Die anderen beiden Aufgabenstellungen konnte man aus einer Liste ausw\"ahlen, oder sich nach Absprache mit den Betreuern eine eigene Aufgabe ausdenken. Wir haben uns als zweite Aufgabe den Rundkurs mit Hindernissen ausgesucht, da wir gehofft haben, f\"ur die erste und zweite Aufgabenstellung denselben Ansatz verwenden zu k\"onnen. Des weiteren haben wir es uns zur Aufgabe gemacht, eine Personenverfolgung zu implementieren. Diese sollte eine Person mithilfe des Kamerabildes detektieren und ihr dann folgen, falls sie sich bewegt.}