\section{Einleitung}
\label{sec:einleitung}
Im Projektseminar Echtzeitsysteme werden die Teilnehmer in Gruppen \'{a} 5 Leute aufgeteilt. Jede Gruppe bekommt ein Modellfahrzeug, welches mit Sensorik und Aktorik so ausgestattet ist, dass das Fahrzeug autonom fahren kann.\\
Das Ziel der Veranstaltung ist es, mithilfe des Autos drei verschiedene Aufgaben zu l\"osen. Die erste Aufgabe wurde dabei vom Veranstalter vorgegeben, die beiden anderen durften die Teams selbst ausw\"ahlen. \\
Die vorgegebene Aufgabe war es, einen Rundkurs ohne Hindernisse entlang einer Wand absolvieren zu k\"onnen. Wir haben dazu zwei verschiedene L\"osungen entwickelt, die beide im Kapitel \nameref{sec:trajektorienplanung} beschrieben werden (siehe Kapitel \ref{sec:trajektorienplanung}).\\
Als zweite Aufgabe haben wir einen Rundkurs mit Hindernissen ausgew\"ahlt. Dieser sollte sich nur anhand der Hindernisse vom ersten Rundkurs unterschieden, deshalb haben wir f\"ur beide Aufgaben den \nameref{subsubsec:02navigatinStack}  verwendet (siehe Kapitel \ref{subsubsec:02navigatinStack}).\\
Unser drittes Ziel war es, eine \nameref{sec:personenverfolgung} zu implementieren (siehe Kapitel \ref{sec:personenverfolgung}). Diese soll eine Person mithilfe des Kamerabildes detektieren und sie anschlie{\ss}end verfolgen, sobald sie sich bewegt.\\
Dieses Dokument beschreibt unsere L\"osungen, unsere Probleme w\"ahrend der Umsetzung und unser Fazit. Zudem gibt es einen Ausblick auf m\"ogliche Weiterentwicklungen f\"ur die jeweiligen Aufgabenstellungen.
