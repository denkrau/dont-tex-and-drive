\section{Fazit und Ausblick}
\label{sec:fazit}
Zu loben ist die freie Gestaltung der Aufgabenstellungen des Projektseminars, was eine Vertiefung der pers�nlichen Interessen erm�glicht. Die Aufgaben wurden gr\"o\ss{}tenteils eigenst\"andig bearbeitet, bei Fragen gab es dennoch immer einen kompetenten Ansprechpartner. Die ausgesuchten Aufgaben der Trajektorienplanung und Personenverfolgung konnten zufriedenstellend gel�st werden, auch wenn h�ufig Hard- und Softwareprobleme auftraten. Darunter z�hlen st�ndige Ausf�lle der \textit{uc\_bridge} und Defekte des Mainboards und eines Hall-Sensors. Unabh�ngig von diesen Problemen w�re es sch�n gewesen, eine Kombination beider Aufgabenteile zu implementieren. Allerdings ist daf�r die Rechenleistung der vorhanden Hardware nicht ausreichend. Ebenfalls war die Zeit daf�r nicht ausreichend und diese Aufgabe wird an interessierte Gruppen weitergegeben. Die Performance des TEB local planner sollte zudem weiter auf Basis der in dieser Arbeit vorgestellten Ans\"atze weiter verbessert werden.