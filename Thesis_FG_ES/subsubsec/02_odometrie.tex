\subsubsection{Odometrie}
\label{subsec:02odom}
Auf dem Fahrzeug befinden sich ein Drei-Achs-Geschwindigkeitssensor und ein Drei-Achs-Gyroskop, welche etwa alle $\SI{5}{\milli\second}$ neue Werte zur Verf\"ugung stellen. Diese werden im vorhandenen Odometriepaket \texttt{pses\_odometrie} genutzt um eine Position und die Ausrichtung des Fahrzeugs anzugeben. Durch das Rauschen der Sensoren unterscheidet sich die tats\"achliche Position  nach einiger Zeit von der angegebenen Fahrzeugposition. Die IMU (inertial measurement unit) wird im Odometriepaket zwar zu Beginn kalibriert, dies bedeutet jedoch nur, dass ein festgelegter, im Stand gemittelter Rauschwert ber\"ucksichtigt und von den IMU Daten subtrahiert wird.\\
Die Verwendung eines Extended Kalmanfilters (EKF) erm\"oglicht die Verwendung verschiedener Sensoren als Eingangsgr\"o{\ss}en f\"ur die Sch\"atzung der Fahrzeugposition und der Ausrichtung. Der EKF berechnet aus dem aktuellen Zustand anhand eines nicht-linearen Zustands\"ubergangsmodells eine Absch\"atzung f\"ur den n\"achsten Fahrzeugzustand. Durch die Verwendung mehrerer Sensoren kann die Genauigkeit und Zuverl\"assigkeit der Sensoren mithilfe einer Gewichtung der Eingangsgr\"o{\ss}en des EKF ber\"ucksichtigt werden.  Die Zustandssch\"atzung wird mit der n\"achsten Messung verglichen und die Gewichtung der Sensoren dynamisch angepasst.\\
Wir haben uns f\"ur das \texttt{robot\_localization} Paket entschieden. Dies enth\"alt einen EKF, das es erm\"oglicht die Sensoren modular einzubinden und sensorspezifisch zu entscheiden, welche Daten weiterverarbeitet werden sollen.
Das Paket kann vier verschiedene Topic-Typen verarbeiten:
\begin{itemize}
	\item \texttt{nav\_msgs/Odometry} enh\"alt Positions- und Ausrichtungsdaten
	\item \texttt{geometry\_msgs/PoseWithCovarianceStamped} enth\"alt Positionsdaten
	\item \texttt{geometry\_msgs/TwistWithCovarianceStamped} enth\"alt Ausrichtungsdaten
	\item \texttt{sensor\_msgs/IMU} enth\"alt die Rohdaten aus der Inertialen Messeinheit (IMU)
\end{itemize}
Um die Sch\"atzung zu vereinfachen wird angenommen, dass sich das Fahrzeug nur im zweidimensionalen Raum befindet und nur in $x$-Richtung fortbewegen kann. $z$-Koordinate, Roll- und Pitch Winkel sind daher konstant Null. Die Beschleunigung in $y$-Richtung wird als Eingangsr\"o{\ss}e vernachl\"assigt. \\
Das \texttt{robot\_localization} Paket erm\"oglicht es, die Kamera in Verbindung mit \hyphenation{AMCL} AMCL, dem Hall-Sensor und der IMU als Eingangsgr\"o{\ss}en f\"ur das EKF zu verwenden. Aus Zeitgr\"unden haben wir das EKF nicht fertig eingestellt, stattdessen verwenden wir f\"ur die Lokalisierung die Odometriedaten in Verbindung mit AMCL, wie im Abschnitt \ref{subsec:02amcl} beschrieben wird. Dies liefert uns eine ausreichend genaue Fahrzeugposition und Ausrichtung.
