\subsubsection{AMCL}
\label{subsec:02amcl}

Zur Lokalisierung des Fahrzeugs wird das von ROS bereitgestellte AMCL-Paket\footnote{http://wiki.ros.org/amcl} verwendet. Dieses Paket implementiert den probabilistischen \emph{Adaptive Monte Carlo Localization} - Algorithmus \cite{amcl} und erm"oglicht dadurch die 2D-Lokalisierung von Robotern in einer vorgegebenen Karte. Die Daten aus dem Laserscan werden auf die Karte projiziert, woraus die wahrscheinlichste Position und Ausrichtung des Fahrzeugs bestimmt wird.\\
Zu diesem Zweck ben"otigt es mehrere Informationen (vergleiche Kapitel \ref{subsec:02implementierung}): 
\begin{itemize}
\item einen Laserscan, der aus dem Tiefenbild erzeugt wird 
\item die initiale Position des Roboters, die mittels  RViz "ubergeben wird
\item die Karte der aktuellen Umgebung
\item die Transformationen zwischen den verschiedenen Koordinatensystemen des Roboters
\end{itemize}
Es sei darauf hinzuweisen, dass AMCL eine explizite Transformation vom Laserscan-Frame (\texttt{base\_laser}) zum Odometrie-Frame (\texttt{odom}) erfordert.
Das AMCL-Paket stellt ebenso eine ganze Menge von einstellbaren Optimierungsparametern bereit. Es ist m"oglich das Odometriemodell (\texttt{omni} bzw. \texttt{diff} f"ur Roboter mit omnidirektionalem bzw. Differentialantrieb) auszuw"ahlen, sowie die entsprechenden Parameter einzustellen. Aus zeitlichen Gr"unden wurde jedoch eine im Navigation Stack bereits vorhandene Launchdatei f"ur differentialgetriebene Fahrzeuge (\texttt{amcl\_diff.launch}) verwendet.