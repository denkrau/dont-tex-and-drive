\subsubsection{Regler}
\label{subsec:03controller}
Die jetzt bekannten Koordinaten $\vec{p} = [p_x, p_y, p_z]^T$ der Person sind aus mehrfachen Gr�nden als finales Ziel ungeeignet. Einerseits ist eine Erkennung der Person schwierig, wenn das Fahrzeug nah an diese heranf�hrt. Andererseits f�hrt es bei station�rer Ausregelung stets die Person an. Daher ergibt sich das finale Ziel bzw. die F�hrungsgr��e aus einer radialen Verschiebung der Personenkoordinaten in Richtung der Kamera mittels des Strahlensatzes zu 
\begin{equation}
	\vec{w} =
	\begin{pmatrix}
		w_x \\w_y \\w_z
	\end{pmatrix}=
	\begin{pmatrix}
		p_x \cdot (1 - d_p / d_{euc})\\ p_y \\ p_z \cdot (1 - d_p / d_{euc})
	\end{pmatrix}
\end{equation}
mit $d_{euc} = \sqrt{p_x^2 + p_z^2}$ und $d_p$ als beliebiger Abstand zur Person. Dabei wird der Abstand zu $d_p = \SI{2.0}{\meter}$ festgelegt, allerdings bietet es sich an, diesen variabel zu gestalten, um zum Beispiel Fahrten um Ecken in schmalen G�ngen zu verbessern.

Eine Anfahrt dieser Zielkoordinate $\vec{w}$ ist auf zwei Varianten m�glich. Die einfachste und ressourcenschonendste ist die Regelung mithilfe eines konventionellen Reglers. Mit der Wahl eines PD-Reglers entspricht diese Variante der hier verwendeten. Dabei muss zum einen der Lenkwinkel mit
\begin{equation}
	\delta = K_{P, \delta} \cdot w_x(t) + K_{D, \delta} \frac{w_x(t) - w_x(t-1)}{dt}
\end{equation}
berechnet werden, um die $x$-Koordinate anzufahren. Zum anderen muss die Geschwindigkeit
\begin{equation}
	v = K_{P, v} \cdot w_z(t) + K_{D, v} \frac{w_z(t) - w_z(t-1)}{dt}
\end{equation}
berechnet werden, um mit einem fl�ssigen Fahrverhalten die $z$-Koordinate zu erreichen. Dabei entsprechen $K_{P, \delta}$ und $K_{D, \delta}$ bzw. $K_{P, v}$ und $K_{D, v}$ den jeweiligen Verst�rkungsfaktoren, $t$ dem Abtastzeitpunkt und $dt$ der Differenz beider Abtastzeitpunkte. Sofern station�re Genauigkeit gew�nscht ist, m�ssen die Regelgesetze um einen I-Anteil erweitert werden und eine Implementierung eines Anti-Windups sollte erfolgen.

Als zweite Variante ist die Kombination mit einer Trajektorienplanung m�glich. Hier kann der Navigation Stack aus Kapitel \ref{subsec:02navigatinStack} bei gleichbleibender Konfiguration verwendet werden. Das Ziel $\vec{w}$ wird stetig als ROS-Message PoseStamped gesendet, welche direkt als Zielvorgabe dem Planer �bergeben werden kann. Somit liegt ein dynamisches Ziel vor, das mit einer Erkennung von m�glichen Hindernissen und einer Umgehung dieser angefahren wird. Hierbei ist allerdings zu beachten, dass diese Variante sehr rechenintensiv und auf der aktuellen Hardwarekonfiguration des Fahrzeugs nicht lauff�hig ist.
